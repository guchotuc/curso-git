\documentclass{sfcursos}

\usepackage{sfestilo}

\NewDocumentCommand{\titulologo}{}%
{
    \fontfamily{LinuxLibertineT-OsF}\selectfont 
    {\textcolor{ggreen}{Git}\ y \textcolor{gblue}{Github} }
}

%%%%%%%%%%%%%%%%%%%%%%%%%%%%%%%
%%% Encabezado del Practico %%%
%%%%%%%%%%%%%%%%%%%%%%%%%%%%%%%
\title{Clase}
\contacto{\faWhatsapp\ (381) 334 6732 / \faEnvelope\ sfoguet@face.unt.edu.ar}

\semester{1er Semestre}
\ciclo{2026}

\duedate{Jueves, 14/08/2025, 09:00hs} % Completar con la fecha de entrega
\assignno{1} % Completar con el nro del TP.
\mainproblem{Configuración Inicial y Línea de Comando} % Completar con el nombre del TP

\begin{document}

\maketitle

\begin{informacion}[colbacktitle=gblue]{\Large Objetivos de la Clase}
Que el estudiante logre
\begin{cirlist}
    \item entender que son los sistemas de control de versión.
    \item instalar el software necesario para raelizar controles de versión.
    \item usar la línea de comando para acceder a archivos.
    \item algo mas que no se que será.
\end{cirlist}

\end{informacion}


\section{Configuración Previa}

\begin{task}{}{}

\begin{itemize}
    \item Ir a la página web:  https://git-scm.com/
    \item Descargar la versión de git correspondiente a su sistema
            operativo. La última versión de git vigente es la $2.51.2$
    \item Ejecutar el archivo descargado y aceptar las opciones por 
            por defecto.
\end{itemize}

\tcblower
\vspace{5cm}
\end{task}

\section{Línea de Comando}

\begin{task}{}{}

hola

\tcblower
\vspace{5cm}
\end{task}


\end{document}